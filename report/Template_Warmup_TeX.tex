\documentclass[12pt,letterpaper]{article}

%----------------------------------------------------------------------
% Standard LaTeX2e packages
%----------------------------------------------------------------------
\usepackage{graphicx}
\usepackage{subfigure}
\usepackage[nospace]{cite}
\usepackage{tabularx}
\usepackage{multirow}
\usepackage{amsmath}
\usepackage{amssymb}
\usepackage{listings}
\usepackage{xurl}
\usepackage{enumitem}
\usepackage{fancyhdr}
\usepackage[ruled,vlined,linesnumbered]{algorithm2e}
\usepackage{wela}
\usepackage[T1]{fontenc}

%---------------------------------------------------------------------------
%   Margin settings
%---------------------------------------------------------------------------
\setlength{\topmargin}{-0.1in}%
\setlength{\headheight}{-0.1in}%
\setlength{\headsep}{0.2in}

\setlength{\footskip}{0.3in}%
\setlength{\textheight}{\paperheight}%
\addtolength{\textheight}{-2.2in}%

\setlength{\oddsidemargin}{-0.2in}%
\setlength{\evensidemargin}{-0.2in}%
\setlength{\textwidth}{\paperwidth}%
\addtolength{\textwidth}{-1.6in}%

% Paths for figures
\graphicspath{{./}{./figures/}}

% Configuration for pseudocode
\SetAlgoHangIndent{0pt}
\SetCommentSty{emph}
\allowdisplaybreaks
\lstset{breaklines=true,basicstyle=\tt, commentstyle=\fontseries{lc}}

%---------------------------------------------------------------------------
%    Style
%---------------------------------------------------------------------------
\fancypagestyle{plain}{%
    \renewcommand{\headrulewidth}{0pt}%
    \fancyhf{}%
    \fancyfoot[R]{\thepage}%
}
\pagestyle{plain}%

\renewcommand{\baselinestretch}{1.15}
\parskip 0pt % sets spacing between paragraphs
\parindent 0pt % sets leading space for paragraphs

\makeatletter
\renewcommand{\theenumi}{\arabic{enumi}.}
\renewcommand{\labelenumi}{\theenumi}
\renewcommand{\p@enumii}{\theenumi.}
\makeatother
\renewcommand{\labelitemi}{$\bullet$}%
\renewcommand{\labelitemii}{--}%

%---------------------------------------------------------------------------
%    Custom Commands
%---------------------------------------------------------------------------

\newcommand{\signature}{\parbox{5cm}{
	\fontfamily{wela}\selectfont
	\centering \Large{Shah} \hrule}
}

%======================================================================
%   Document begins here
%======================================================================
\begin{document}

\begin{center}
    \Large \textbf{CS 3873:  Net-Centric Computing}
\end{center}

\begin{center}
    \Large \textbf{Lab Exercise 1:  Preparation and Warm-up} \vspace*{24pt}
\end{center}

Student Name:
	\parbox{5cm}{\centering Shahriar Kariman \hrule}
Student Number: 
	\parbox{3cm}{\centering 3665643 \hrule}
\\

\textbf{[Mandatory]} Declaration: ``I warrant that this is my own work.'' 
\\

Signed by \signature
\\

Leave the table blank. For marking only.
%------------------------------------------------------------------------
\begin{table}[!h]
\renewcommand{\arraystretch}{1.3}
\centering
\begin{tabular}{|>{\centering}m{25mm}|m{30mm}|m{35mm}|m{35mm}||m{25mm}|}
\hline
\bf{Achieved} & \bf{Late Penalty} & \bf{Others (Specify)} & \bf{Others (Specify)} & \bf{Total Score}\\
\hline
 & & & &  ~~~~~~~ / ~~~~ \\
\hline
\multicolumn{5}{|l|}{Additional comments if any:} \\
\multicolumn{5}{|l|}{} \\
\multicolumn{5}{|l|}{} \\
\multicolumn{5}{|l|}{} \\
\multicolumn{5}{|l|}{} \\
\multicolumn{5}{|l|}{} \\
\hline
\end{tabular}
\end{table}
%------------------------------------------------------------------------

From the UNB Undergraduate Calendar, available at the website: \url{http://www.unb.ca/academics/calendar/undergraduate/current/regulations/universitywideacademicregulations/viii-academicoffences/index.html.}

``Plagiarism includes:
\begin{enumerate}[itemsep=0pt]
\item[1.] quoting verbatim or almost verbatim from any source, regardless of format, without acknowledgement;
\item[2.] adopting someone else's line of thought, argument, arrangement, or supporting evidence (such as, statistics, bibliographies, etc.) without indicating such dependence;
\item[3.] submitting someone else's work, in whatever form (essay, film, workbook, artwork, computer materials, etc.) without acknowledgement;
\item[4.] knowingly representing as one's own work any idea of another.''
\end{enumerate}

Penalties for plagiarism and other academic offences can be found at the above website.



%======================================================================
\newpage
\begin{center}
    \Large \bf Report for Lab Exercise 1:\\Preparation and Warm-up \vspace*{12pt}
\end{center}


{\bf LAB ACTIVITIES:}
\\

In this lab, we learnt how to capture packet traces with Wireshark and how to use Wireshark to examine
the details of captured packets.
\\

{\bf ANSWERS TO LAB QUESTIONS:}

\begin{enumerate}
    \item[6.] Based on the above trace, answer the following questions.
    \begin{enumerate}
        \item[a.] Select the first HTTP message shown in the packet-listing window. This should be the HTTP GET
message that was sent from your computer to the HTTP server. When you select the HTTP GET
message, the Ethernet or Ethernet II frame, IPv4 datagram, TCP segment, and HTTP message
header information will be displayed in the packet-header window. How long did it take from
when the HTTP GET message was sent until the first HTTP response was received?

        {\bf Answer:} The GET message was captured at approximately $4.24_s$ and the response was captured at $4.31_s$ that means it took $0.07_s$ to get a reply.
    \end{enumerate}
\end{enumerate}



%==================================================================
\end{document}
%==================================================================


